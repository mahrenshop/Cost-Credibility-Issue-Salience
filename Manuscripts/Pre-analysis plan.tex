\documentclass[11pt]{article}
\usepackage{setspace}
\usepackage{amsmath}
\usepackage[utf8]{inputenc}
\usepackage[english]{babel}
\usepackage{listings}
\usepackage{geometry}
\usepackage{scrpage2}
\usepackage[colorinlistoftodos]{todonotes}
\usepackage{lmodern}
\usepackage{booktabs}
\usepackage{makecell}
\usepackage{tabularx}
\usepackage{graphicx}
\usepackage{hyperref}
\usepackage{float}
\usepackage[nottoc,notlot,notlof]{tocbibind}
\usepackage{authblk}
\usepackage{setspace}

\title{Costs, Credibility, and Relative Salience: \\ Why Voters Don't Sanction Politicians for Corruption}
\author{MATS AHRENSHOP\footnote{Graduate Student at Department of Politics and International Relations, University of Oxford.}}

\date{This version: \today \\ Prepared for Presentation at the 2018 Toronto Political Behaviour Workshop}

\begin{document}

\maketitle

\begin{onehalfspace}
\begin{center}
\textbf{Abstract}
\end{center}

\noindent Why don’t voters punish corruption in elections under optimal informational conditions? Retrospective voting models argue that informational deficits distort a perfect principal-agent relationship between voters and elected officials. However, even if information about corrupt politicians is existent and credible, retrospective punishment of corruption is often non-existent.
I apply a different theoretical perspective to corruption voting by combining prospective cost evaluations with the relative issue-salience in a formal model. I thus hypothesize that voters do not punish corruption because corruption is less salient to voters than other issues such as prosperity and security. I argue that this is because voters are not willing to tolerate the costs associated with fighting corruption if other issues are more salient -- especially when anti-corruption policy promises are non-credible.
I isolate relative-issue salience as the causal mechanism linking anticipated costs and promise credibility of the promise to vote choice in two steps. First, I will conduct a survey experiment treating U.S. subjects with vignettes on costs and credibility of policy proposals and compare the relative issue-salience of corruption, prosperity and security. Second, I will employ a conjoint experiment where U.S. voters choose between two candidates promising policies with randomly varying levels of policy issue, costs, and credibility.
\end{onehalfspace}

\newpage
\doublespacing

\section{Introduction}
Why don't voters punish corruption in elections under optimal informational conditions? Retrospective voting models assume that there is a cap in the connection between information acquisition, responsibility attribution, and vote choice due to informational deficits. This distorts a perfect principal-agent relationship between citizens and elected officials. However, (i) evidence here is often drawn from observational data, making causal inferences hard to achieve, and (ii) there is evidence that even if information about corrupt politicians is existent and credible, retrospective punishment is often absent. Why is the latter the case?

I combine prospective voting models and relative issue-salience and add the argument to the corruption voting literature that voters do not punish corruption at the ballot box even under optimal information scenarios because corruption is less salient to voters than other issues such as the economy and domestic security. I argue that this is because voters are not prospectively willing to tolerate the costs of fighting corruption, especially when anti-corruption policy promises are not credible, whereas for relatively more salient issues such as prosperity and greater domestic security they are willing to tolerate costs. Explanations of the corruption-voting puzzle derived from prospective voting, as well as the relative issue-salience of corruption are largely absent from the corruption voting literature, as are policy campaign promises as signal extraction cues within the prospective voting literature. I attempt to help close this gap.

My argument will be tested by one standard survey experiment estimating relative issue-salience and by one conjoint experiment where voters have to choose between two candidates offering policies with randomly varying levels of the attributes costs, credibility, and the issues itself.

The research question of why voters often fail to punish corruption has, first, important implications for optimal campaign platform design, as the conjoint experiment models the process of voters choosing between candidates competing on different policy packages: What would voters be willing to tolerate? Second, providing evidence on the identified puzzle has implications for electoral accountability in general and for democratic self-governance which lie at the core of any democratic system.

\newpage
\section{Literature Review}
How elections can yield representation and accountability is of utmost concern to any democratic society. However, it is not always clear whether elections can live up to their promise of optimally reconciling citizens' interest with public policy and holding elected officials accountable (Przeworski 2018). Under a perfect scenario, basic accountability models of retrospective voting have looked at the relation between citizens and elected officials as a principal-agent relationship, as in early models by Key (1966) and Barro (1973), and later Fiorina (1981) and Ferejohn (1986). These contend that voters are able to evaluate the incumbent's performance of the past election cycle and sanction bad performance in office or reward good performance by voting against or for them, respectively, as a solution to moral hazard (Healy and Malhotra 2013).

Some empirical studies have been able to confirm this accountability mechanism as it works in punishing and rewarding corrupt political candidates. For example, Ferraz and Finan (2008) find that information about corrupt candidates in the form of publicly released audit reports in Brazil provide material for voters' performance evaluation and hence find strong electoral accountability effects. Relatedly, they show that re-election incentives decrease the level of corruption in local governments (Ferraz and Finan 2011). Krause and Méndez (2009) provide further evidence on successful corruption punishment at the ballot box.

However, numerous other empirical studies have not been able to show these retrospective voting effects when it comes to corruption, as de Vries and Solaz (2017) conclude in their review. Answers as to why this might be the case have largely blamed the failure of one or more mechanisms within the retrospective voting model. In a cognitive theoretical tradition, Manin et al. 1999 argue that informational asymmetry distorts the assumed perfect principal-agent relationship between citizens and elected officials when voters simply do not know about the corrupt activities of their agents (Chang et al. 2010) or when their information is biased due to in-group loyalties (Anduiza et al. 2013; Charron and Bagenholm 2016). A second, institutional, variant holds that in some cases there is no clear responsibility for corrupt behavior for one party or another (Tavits 2007; Schleiter and Voznaya 2014; Schleiter and Voznaya 2016); or, in a diametrically opposed scenario, voters do not know whom to \textit{reward} when there are no “clean" alternatives (Eggers 2014; Eggers and Spirling 2017). Yet another variation is that even if there is information, it is not regarded as credible by voters, as Winters and Weitz-Shapiro (2013) as well as Botero et al. (2015) have shown. Furthermore, a final, more structural, explanation holds that under a scenario with high levels of corruption, it might be entirely financially rational for voters to tolerate corruption (vote buying; Manzetti and Wilson 2007; Klasnja et al. 2016).

All four reasons for why accountability models would predict a failure to electorally sanction corrupt politicians -- informational asymmetry, flawed responsibility attribution, non-credibility of information, and vote buying -- have been inconclusive. There are several reasons for this. First, many studies draw evidence from observational survey data while making causal claims, thus facing endogeneity and unobserved heterogeneity problems (e.g., Peters and Welch 1980; Chang et al. 2010). Second, the accumulated empirical evidence has not been conclusive: It has been shown that even in cases where information is existent, when responsibility attribution is relatively straightforward, and when information is credible, retrospective voting nevertheless is often non-existent (de Vries and Solaz 2017, 197). Third, the only explanations looking at \textit{costs} voters have to incur when punishing corrupt politicians are in the vote buying literature which almost entirely studies high-corruption countries.\footnote{In these countries,`scandal fatigue' and informational deficits are valid explanations for the corruption-voting puzzle (Klasnja and Tucker 2013; Klasnja et al. 2018).}

Why the lack of accountability still persists in \textit{low}-corruption countries is thus the central puzzle of this research project. Which causal mechanism, if not retrospective voting, is it that can account for persistent corruption voting? I offer two types of contributions to this problem. First, I try to fill the gap methodologically, by piling up the stock of experimental evidence. De Vries and Solaz (2017) provide one of the rare overviews on electoral consequences of corruption which also reviews experimental and quasi-experimental studies; the few experimental studies on the issue still do not mount up to a conclusive body of causally credible empirical evidence which can travel across time and space, casting doubt on the external validity of these experimental findings. Testing new theoretical arguments in different settings and contexts using experimental methods contributes to this larger research program of causal knowledge accumulation (Samii 2016; Green and Gerber 2012; Angrist and Pischke 2010).

Second, substantially, I add a new theoretical argument to the puzzle identified above. The retrospective voting explanations for why the accountability mechanism fails to apply in the corruption case are not entirely conclusive and causally persuasive. \textit{Prospective} voting explanations for the failure of electoral accountability as regards corruption have been largely absent in the literature reviewed, as well as relative issue salience within prospective voting models in general, have, with a few exceptions (Chang et al. 2010; Klasnja et al. 2016), been neglected. Furthermore, anti-corruption policy-promises made during campaigns instead of past performance as signal extraction cues for prospective vote choice have not been seriously considered in the general selection literature, neither more recently in Ashworth (2012) and Duch and Stevenson (2008), nor in Fearon (1999).

Hence, I offer a new explanation, combining prospective voting with a formal model of issue salience, thus bridging the gap between rational choice and political psychology approaches. My argument explicitly does not contend that the steps from information acquisition to correct responsibility attribution to vote choice in the retrospective voting model are simply too large for voters to manage, as de Vries and Solaz (2017) suggest, but that corruption simply is not a salient issue to voters relatively ranked to other issues such as economic prosperity and domestic security. I argue that this is a function of anticipated costs of and lack of credible commitment in policy promises as voters look to the future rather than to the past in evaluating candidates' likely performance. How exactly, then, does issue-salience work as a mechanism explaining persistent corruption voting?


\newpage
\section{Theoretical Argument}
Combining the previously largely neglected approaches prospective voting with relative issue-salience, I model two main theoretical processes, as displayed in \autoref{fig:fig1}. First, traditional prospective voting (Lockerbie 2008) contend that voters vote for the party they believe will deliver the best performance in the future to maximize individual welfare by using campaign promises (their anticipated costs and credibility) as signal extraction cues for vote choice. Second, however, as we know from the issue-salience literature (Miller et al. 2016), not only vote choice itself, but personal issue-salience is affected by material self-interests as well. Thus, regarding relative issue-salience as the causal mechanism linking anticipated costs and credible commitment to vote choice is a logical step. The central argument is that \textit{voters do not punish corruption in elections because corruption is less salient to voters than other issues such as the economy and domestic security. I argue that this is because voters are not willing to tolerate the costs of fighting corruption, especially when anti-corruption policy promises are not credible, whereas for relatively more salient issues such as prosperity and greater domestic security they are willing to tolerate costs.}

\subsection{Traditional Prospective Voting}
In the probabilistic prospective voting model, voters form their expectations based on promises made by political parties or candidates during an election campaign. I adopt and simplify the model by Elinder et al. (2015) by excluding their retrospective elements and adjust it to the theoretical approach of this project.

Consider two parties, $L$ and $R$. The outcome of interest is the likelihood that citizen $j$ votes for party $L$. The game is being played as follows: Political parties present their platforms (campaign promises) at the beginning of the period, i.e. during an election campaign. Thereafter, citizens vote and the winner forms the government. We can denote the expected benefits of citizen $j$ if party $K, K \in \{L, R\}$ would win as $E[B_j^K]$. In the simplest of all scenarios, voter $j$ votes for party $L$ if

\begin{equation}
E\bigg[B_j^L - B_j^R\bigg] > 0.
\end{equation}
Pure prospective voting models thus contend that the expected benefits if party $K$ would win office are entirely a function of party $K's$ promised benefits to citizens, $P_j^K$. Thus,

\begin{equation}
E\bigg[B_j^K\bigg] = P_j^K.
\end{equation}
Inserting (2) into (1) yields the following condition for citizen $j$ voting for party $L$ if

\begin{equation}
P_j^L - P_j^R > 0.
\end{equation}
Put informally, in order to maximize individual financial benefits, voters select the party under which they expect to fare best financially in the future. To calculate expected utility under the new adminstration, “voters have good reason to pay attention to the promises of candidates," as Lockerbie (2008, 7) states. It is crucial to note, however, that this “pure" model assumes perfect credibility of party promises because citizens are assumed to vote entirely based on promises. Weights for relative issue-salience are also not considered here.

\subsection{Modified Prospective Voting}
\paragraph{Outcome}
The main aim is to credibly estimate the relative issue-salience of corruption and then to use this as an argument for why voters do not electorally punish corrupt politicians. To conceptualize the first outcome of interest, (relative) issue-salience, let us examine the deterministic spatial model of voting in Ansolabehere and Puy (2018). Consider political party $j$ with platform $(x_j, y_j)$ for issues $X$ and $Y$, and voter $i$ with ideal policy $(x_i, y_i)$. The preference of voters over political parties, $U_i(j)$, is the negative quadratic distance between the party platform and the ideal point on each issue,

\begin{equation}
U_i(j) = - \alpha(x_j - x_i)^2 - \beta(y_j - y_i)^2,
\end{equation}
where $\alpha, \beta > 0$ are \textit{salience parameters}, i.e. weights of general importance voters assign to issues $X$ and $Y$\footnote{Note that this is the most common definition of issue-salience and is in line with Miller et al.'s (2016) definition. For other conceptualizations, see Miller et al. 2016, 125-130.}. It is sufficient to note that these salience weights become politically consequential when voters' optimal voting decision is cast upon the differential utility between parties A and B, $\Delta u_i = U_i(A) - U_i(B)$.

The important point in Ansolabehere's model, however, is the concept of relative issue-salience (RIS),
\begin{equation}
RIS = \bigg[\frac{\alpha}{\beta}\bigg]^\frac{1}{2}.
\end{equation} 
This yields the average importance in voters' preferences of issue $X$ over issue $Y$ and constitutes the first outcome of interest.

To conceptualize the second outcome of interest, vote choice, as needed for the “larger" prospective voting model, I simply consider the choice of voting for candidate $A$ as opposed to voting for candidate $B$, where the optimal decision rule of selecting a candidate is the same as in equation (3), but this time for candidates instead of parties.

\paragraph{Treatment}
How can we explain the relative issue-salience of corruption? Miller et al. (2016) identify three different sources affecting personal issue importance: Material self-interest, values, and identification. First, looking at ``material self-interest'', it is crucial to note that costs are allowed to vary. They need not solely be financial costs as in the pure prospective voting model, but for example can also be costs such as a loss of privacy (handing in data for a newly proposed anti-corruption commission). This can be applied to corruption voting and the anticipated costs of anti-corruption campaign promises and we can thus hypothesize that \\
\textit{H1: Costs associated with anti-corruption policy promises decrease the relative weight voters place on corruption.}

Second, Elinder et al. (2015) -- in line with Feltovich and Giovannoni (2015) -- argue that “in a richer model, $\gamma_{m, t}$ [weights voters attach to platform of party K] would be endogenous, and arguably depend on the extent to which promises have been kept in the past." Applied to the corruption voting case, we can hypothesize that \\
\textit{H2: Lack of credibility of anti-corruption policy promises decreases the relative weight voters place on corruption.}

Third, Miller et al. (2016) consider “values" as a major source of personal issue-salience. I thus expect that for other issues that are higher valued by voters, such as the economy and domestic security, costs and lack of credibility of respective policy proposals do not affect their relative issue-salience, since voters are willing to incur costs for a better economy/more security. The economy and security concerns have consistently been shown to be two of the most important issues to voters (Gallup News 2018; Aisch and Parlapiano 2017). Thus, \\
\textit{H3: Anticipated costs of policy promises and lack of perceived credibility do not affect the relative issue-salience of the economy and security.}

Finally, to close the causal chain, I apply this “zoomed-in" framework, where relative issue-salience is the outcome, to the broader model of prospective voting, thereby linking it to vote choice as the ultimate outcome. Formally, the pure prospective voting model derived above will simply be extended, namely, if candidate $K$ campaigns on anti-corruption promises (among others), then

\begin{equation}
E\Big[B_j^K\Big] = \sum_{i = 1}^{3}\bigg(P_j^{K, i} \times \rho_i + RIS_{j,i}\bigg),
\end{equation}
where $\sum_{i = 1}^{3}P_j^{K, i}$ designates the promised benefits (anticipated costs) for voter $j$ made by candidate $K$ for each issue $i = 1, \hdots, 3$, $\rho_i$ is the credibility of the proposal for issue $i$, and $RIS_{j,i}$ is the relative issue-salience voter $j$ attaches to issue $i$ within the proposal, where for each issue there are two comparison weights, and thus two $RIS_{j,i}$ which are averaged for each issue. The decision rule for the optimal strategy to vote is still captured in equation (1), but now the expected benefits are furthermore a function of anticipated costs of the promise and credibility of the promise \textit{plus} the relative issue salience of the issue within the promise.

From this, we can informally hypothesize that \\
\textit{H4: Lower salience of corruption decreases the probability of voting for a candidate campaigning on anti-corruption policies.} \\
To sum up, voters do not punish corruption in elections because corruption is less salient to voters than other issues such as the economy and domestic security. I argue that this is because voters are not willing to tolerate the costs of fighting corruption, especially when anti-corruption policy promises are not credible, whereas for relatively more salient issues such as prosperity and greater domestic security they are willing to tolerate costs. By looking at the two causal processes separately, where relative issue-salience is, only conceptually, first treated as the outcome and then as an independent variable affecting vote choice, I attempt to isolate the causal mechanism between policy promises and vote choice: relative issue-salience.\footnote{There is grounds for concerns about post-treatment bias of the relative issue-salience coefficient, potentially violating the conditional independence assumption. First, this will be addressed by balance checks within treatment groups, see Acharya et al. 2016. Second, the two processes will be estimated in separate experiments with different samples, thus technically, the coefficient on the relative issue salience from the first model will not be used in the second model.}

In line with Miller et al.'s (2016) ``identification'' as a third source of relative issue-salience, I relax the implicit rationality assumptions from the models and allow for heterogeneous treatment effects: Depending on party identification (PID; Arceneaux and Johnson 2013), the effect of anticipated costs and lack of credibility of policy promises on the relative issue-salience can be quite different. I also expect voters' responses to anticipated costs of promises and a lack of credibility to be especially strong among low-income citizens.

%--------------------%
\begin{figure}[!ht]
\begin{center}
\includegraphics[scale = 0.45]{causalmodel2.JPG}
\caption{Causal Model}
\label{fig:fig1}
\end{center}
\end{figure}
%--------------------%

\newpage
\section{Case Selection and Data Gathering}
\subsection{Case Selection and Sampling Strategy}
My case selection is based on two criteria to delimit the case universe. First, and most obviously, the experiments will be conducted in an advanced democracy holding free and fair elections. The point here is to have a setup where policy promises made during election campaigns, the central treatment of these experiments, are meaningful, i.e. they \textit{theoretically} can have an effect on the relative importance citizens place on the issues under considerations as well as on their vote choice. Meaningful electoral competitions do not exist in authoritarian regimes, and thus there is no reason for voters to think that these promises could be politically consequential. In full democracies, however, we can expect that campaign promises can induce, at least theoretically, some variation in relative-issue salience and vote choice, because voters can treat them as meaningful signal extraction cues needed in the selection voting model, whereas promises in authoritarian regimes (and hybrid regimes) do not fulfill this function.

Second, the experiments will be conducted in a low-corruption country, because if the Corruption Perceptions Index (CPI, Transparency International 2018) for a given country is low, then politicizing the issue, e.g. through campaign promises, should have a noticeable effect on experimental subjects and their response to the treatments, as opposed to high-corruption countries with “scandal fatigue" (Klasnja et al. 2016; Klasnja and Tucker 2013), where an additional politicization of the issue could not easily be regarded as a real “treatment". It is doubtful whether in these high-corruption countries there is any room for the relative-issue salience of corruption (as an outcome) to vary at all, simply because of corruption being an everyday reality. In sum, ideally, concepts and findings could “travel" to every country within this realm of low-corruption advanced democracies. However, a note of caution is in order as to the fact that institutional features vary across countries within this group, decreasing the generalizability of the inferences we can make from these experiments. Either of two countries, the United States or the United Kingdom, would provide interesting cases since their two-party system closely approximates the ideal situation formalized in the modified prospective voting model; plus, they are low-corruption countries as measured by CPI. The choice between the two, however, ultimately depends on funding opportunities for the experiments. Yet, both countries would provide interesting cases; the U.K. is ranked in the top quartile of CPI (rank 8) and would serve as the ``bookend" case; if the priming of the corruption issue worked here, this could easily be generalized to countries with even lower levels of perceived corruption. The U.S. is ranked rather in the middle (rank 16), which would justify generalizability of the findings in either direction of the corruption level.

Online labor markets such as Amazon.com's Mechanical Turk (MTurk) or YouGov are typically used for survey experimental subject recruitment. These represent convenience samples, i.e. participants opt into the panels voluntarily or for monetary incentives and thus become experimental subjects. Thus, the external validity of this sampling strategy has been questioned on theoretical grounds and discarded empirically (Berinsky et al. 2012; Huff and Tingley 2015) in comparison to national probability samples. This is logical since there is no randomization of subject \textit{recruitment} involved here.

However, MTurk provides a low-cost and easy-to-field opportunity to run survey experiments with reasonably high internal validity of the convenience sample. Berinsky et al. (2012) show that these are often more representative of the U.S. population than in-person convenience samples; Duch et al. (2016) find only mixed evidence for covariate imbalance comparing online convenience and population-based samples. Similarly, Coppock (2018) shows that despite differing background characteristics across samples obtained from population-based samples and MTurk, 15 replications experiments show that results derived from convenience samples yield similar results to those derived from national samples. Due to these advantages, and considering the specificities of this project, an online-survey experiment is preferred over a lab experiment. Furthermore, an ideal case of obtaining a probability sample from a nationally representative sampling frame and then conducting the treatment randomization procedure is not feasible in this project, largely due to high costs and time effort.

\subsection{Data Gathering: The Experimental Design}
The puzzle of why voters do not punish corruption in elections under optimal informational conditions entails, I argue, two “effects of causes"-questions: \\
(1) What is the effect of anti-corruption policy proposals on the relative issue-salience of corruption?\\
(2) What is the effect of the relative issue-salience of corruption on vote choice?\\
The two questions require two separate identification strategies to credibly estimate the causal effects of each of the respective treatment variables; without a credible identification strategy, one would face serious selection bias. There will always be a lingering question about which covariates have not been modeled, either forgotten or because they would be technically unmeasurable (Samii 2016; Keele 2015). Thus, in using experimental data obtained from two separate survey experiments, we can eliminate selection bias and unobserved heterogeneity since with random assignment to treatment and control groups, we can credibly model a counterfactual situation to answer the question of what would have happened to the outcome had the treatment not been applied. That is, if the assumption about the independence of potential outcomes of treatment assignment holds by virtue of random assignment as in

\begin{equation}
Y_{1i}, Y_{0i} \perp D_i,
\end{equation}
then, by design, treatment and control groups are, in expectation, similar in all observed \textit{and} unobserved background characteristics (Gerber and Green 2012; Angrist and Pischke 2009; Holland 1986), i.e.

\begin{equation}
E[Y_i | D_i = 1] = E[Y_i | D_i = 0].
\end{equation}

As noted above, this proposed two-step estimation approach to isolate the causal mechanism provides a valuable strategy to counter the fact that the mediator, relative-issue salience, is usually not randomly assigned. Furthermore, I employ the design of complex treatments to answer the question: Which part of the treatment is really doing the work here (Imai et al. 2011; Baird et al. 2011)? I collapse the treatment into multiple treatment arms: combinations of anticipated costs and credibility of the proposal in the first experiment; simultaneous randomization of several treatment components in the second conjoint experiment.\footnote{The two experiments will be pre-registered on \href{http://egap.org/}{EGAP} to make use of the several transparency and replicability advantages pre-registration offers. Power calculations will be performed there.}

\paragraph{Salience estimation approach}
As hypothesized, anticipated costs and lack of credibility of anti-corruption policy promises decrease the relative weight voters place on corruption, whereas they do not affect the relative issue-salience of the economy and security (hypotheses 1-3). To test these hypotheses, I employ a first survey experiment with the following experimental set-up. The two main treatment groups with two treatment arms, respectively, are a combination of the two factors anticipated costs and credibility of the policy-promise. Note that for each of the four resulting treatment arms, each subject receives its respective treatment for all three issues -- corruption, economy, domestic security -- consecutively; these policy bundles are much closer to the political reality of electoral campaigns. To avoid contamination effects, the order in which subjects within each treatment arm receive the issue-proposals will be randomized within each treatment arm, as displayed in \autoref{fig:fig2}.

The first group receives informational interventions about a credible but costly policy proposal made by a candidate during an election campaign. The second group is treated with a costly policy proposal, without mentioning credibility of that proposal at all. The third group receives an informational intervention about a non-credible and costly policy promise. The fourth treatment group receives information about a non-credible policy proposal, without mentioning anticipated costs of that proposal at all. Finally, the control group receives information about the promised proposal for each respective issue, without any mention of anticipated costs or credibility. A placebo group will receive a text about Hilary Hahn's latest concert in Carnegie Hall, which is assumed to be completely unrelated to the outcome, relative issue-salience. Comparing this outcome across the different treatment arms as well as between each treatment arm and the control/placebo group allows the credible estimation of the treatment effects proposed in hypotheses 1-3.

A baseline survey (Broockman et al. 2017) will be run to collect information on supposedly prognostic pre-treatment covariates, in this project used for increasing precision in the estimation approach, rescaling the outcome variable as well as detecting heterogeneous treatment effects (Gerber and Green 2012). A pilot test aims at assessing whether treatments (especially the credibility treatment) and question wording actually work and whether a placebo group will be used, based on power analysis and $ATE's$.

To transform the core concepts derived in the theory section into observable characteristics, i.e. give them operational definitions and make them measurable, let us look at the treatment arms first. The informational interventions here consist of different combinations of anticipated costs\footnote{Ideally, I would allow for the type of costs to vary. However, this would increase the number of different treatment arms such that sample size would be limited. Yet, in the conjoint experiment, types of costs are allowed to vary.} and credibility of policy promises made during election campaigns by a candidate, for each of the three issues corruption, economy, and domestic security, respectively. Experimental interventions are in Appendix 7.1. After receiving treatment, there are different options to measure relative issue-salience.\footnote{All exact question wordings are provided in Appendix 7.2 and 7.3.} Following Broockman et al.'s (2017) recommendation of using multiple measures of outcomes, I employ the following measurement strategies. First, participants are simply asked to rank the three core issues along with three unrelated policy issues according to their relative personal issue-salience, where the order of the appearance of the issues is randomized to avoid contamination effects.\footnote{Three more issues than only the issues of the experimental treatment are added such that respondents have a greater likelihood of finding their personally most important topic in the list and giving thus more valid answers in the ranking.} Another strategy will be to ask respondents to scale the relative importance of each issue.

Regarding the measurement of prognostic pre-treatment covariates, I include several survey questions in the baseline survey. Partisanship is understood in terms of the standard and widely used concept of party identification (PID; Green et al. 2004) and will be measured in three steps, following Lupu (2012, 6): whether participants identify with a political party, and if so, with which and how strongly. Furthermore, respondents are asked to indicate their household income, age, sex, and race, all of which are standard control variables in survey panels.

%--------------------%
\begin{figure}[H]
\centering
\includegraphics[scale = 0.35]{expsetup2.JPG}
\caption{Experimental Design 1 and Treatment Effect Expectations}
\label{fig:fig2}
\end{figure}
%--------------------%

\paragraph{Vote choice estimation approach}
To close the causal chain, hypothesis 4, which states that low relative issue-salience of corruption decreases the probability of voting for a candidate campaigning on anti-corruption policy promises, is evaluated using experimental data from a conjoint experiment. This special form of survey experiment is the perfect fit for the question at hand, both as it relates to the proposed explanatory variables as well as to the outcome. Hainmueller et al. (2013) provide us with a formal framework and a statistical estimation approach to estimate the causal effects of simultaneously varied multiple treatment components (multidimensional treatments), thus being able to assess which components of the manipulation produce the observed treatment effects on the outcome. This is ideal when the goal is to isolate causal mechanisms and thus captures exactly the complexity of the treatment as derived above. Second, the authors provide an excellent framework for estimating these marginal component effects on multidimensional decision-making processes, as they occur in voting behavior all the time. In a conjoint design, respondents are asked to choose from a set of hypothetical profiles that combine multiple attributes in what is called a “forced-choice" design. In sum, this approach is more realistic to the multidimensionality of (i) electoral decision-making processes (choosing between $\geq 2$ candidates) and (ii) the multidimensionality of treatments (policies consisting of $\geq 2$ elements) than standard survey experiments are.

The experimental setup is as follows. For each of several choice tasks, respondents are presented with two candidates proposing policy-packages (2 profiles) composed of different attributes. According to the formal model developed in the theory section, I will focus on the following three attributes: costs with the levels \{public money (taxes), opportunity costs, personal data, no costs\}, credibility with the levels \{credibility, no credibility\}, and issue with the levels \{corruption, economy, security\}. In each policy-package (profile), a level is randomly selected for each attribute from the set of possible levels. Under this randomization scenario, combined with a set of consistency assumptions outlined by Hainmueller et al. (2013), the average marginal component effect of each treatment component is non-parametrically identified. Note that in order for this to work, “the respondents need not be shown every potential combination of profiles/attributes to identify these component-specific effects" (Strezhnev et al. 2014, 2).\footnote{Another nice advantage of this design is obvious here: One does not need a sample size large enough to present respondents with every possible combination.} Thus, in each of the six choice tasks each respondent receives, the respondent is asked to choose between two candidates (profiles) offering different policy-packages from round to round. In line with the recommendation of Broockman et al. (2017) on multiple measurements for outcomes, respondents are also asked to rate the profiles on a one to seven scale for the level of absolute support or opposition to each profile. Thus, relative issue-salience, by virtue of randomly priming it (through alterations in “issue"), salience becomes a coefficient within a broader model of vote choice.

Measurement of the treatment is more straightforward in this case, because we can build upon the vignettes developed for the first experiment. For each choice task, a respondent is asked to read a tabular overview of the two profiles between which to choose (an example is provided in \autoref{tab:tab2} in Appendix 7.4); the information on each attribute here are extracted from the previous vignettes. The outcome will be measured in two ways, following Hainmueller et al. (2013). First, under the “forced choice" scenario, respondents simply have to electorally choose between the two candidates. Second, respondents have to provide a rating of support for each candidate. I include survey questions on prognostic pre-treatment covariates in the baseline survey in the same manner as in experiment 1 (see Appendix 7.5 for exact question wording).

\newpage
\section{Data Analysis} \label{sec:sec5}
\subsection{Experiment 1: Estimating Relative Issue-Salience}
The causal quantity of interest in this first standard survey experiment is the average treatment effect $ATE$ which is simply the expected difference of individual unit's potential outcomes in the treatment and control conditions, formally

\begin{equation}
\widehat{ATE} \equiv \frac{1}{n} \sum_{i=1}^{n} \big[Y_{1i} - Y_{0i}\big].
\end{equation}
By design, we can credibly assume independence of treatment assignment and potential outcomes, and thus estimate $ATE$ with a simple difference-in-group-means estimator,

\begin{align}
\begin{split}
\widehat{ATE} & = E[Y_{1i} - Y_{0i}] \\
& = E[Y_{1i}] - E[Y_{0i}] \\
& = E[Y_i|D_i = 1] - E[Y_i|D_i = 0].
\end{split}
\end{align}
To compare relative issue-salience as the outcome across the different treatment arms and estimate $ATE$ for this first experiment, I will fit an ordered-logit model which is detailed in Appendix 7.6.

Furthermore, by interacting the treatment with PID as well as with income we can model possible heterogeneous treatment effects. Validity checks such as balance tests will be conducted to detect possible covariate imbalance (Gerber and Green 2012) across treatment groups by regressing the different treatment groups on all covariates, respectively, and look for significant effects which would indicate covariate imbalance.

\subsection{Experiment 2: Estimating Vote Choice}
Non-parametrical estimation of the causal quantity of interest in a conjoint experiment, the average marginal component effect $AMCE$, is straightforward under the framework Hainmueller et al. (2013) present given some assumptions are met. Some notation is useful in understanding both the assumptions in place as well as the estimation procedure. Denote by $i \in \{1,...,N\}$ each respondent's index presented with $K$ choice tasks, where in each of these tasks she chooses the most preferred of $J$ profiles. Each of these profiles, in turn, consists of $L$ attributes, where $D_l$ indicates the total number of levels for attribute $l$. Then, $T_{ijk}$ denotes treatment given to respondent $i$ as the $j$th profile in her $k$th choice task. Further assume $T_{ik}$ denotes the entire set of attribute values for all $J$ profiles in respondent $i$'s choice task $k$, and $\bar{T}_i$ the entire set of all $JK$ profiles respondent $i$ sees throughout the experiment.

This can be expanded to the potential outcomes framework, where $Y_{ik}(\bar{t})$ denotes the $J$-dimensional vector of potential outcomes for respondent $i$ in her $k$th choice task that would be observed when the respondent received the sequence of profile attributes represented by $\bar{t}$, with individual components $Y_{ijk}(\bar{t})$. Due to high complexity, the estimation procedure will be presented without further notation (see Hainmueller et al. 2013, 7-17, for full notation). The causal quantity of interest here is the average marginal component effect $AMCE$, which indicates how different values of the $l$th attribute of profile $j$ influence the probability that the profile is chosen; we obtain the marginal effect of attribute $l$ averaged over the joint distribution of the remaining attributes. Put differently, it represents the “increase in the population probability that a profile would be chosen if the value of its $l$th component were changed from $t_0$ to $t_1$ [a pair of profile sets], averaged over all the possible values of the other components given the joint distribution of the profile attributes $p(t)$." (ibid. 2013, 11)

Satisfaction of the conditions outlined in Appendix 7.7 gives us leverage to estimate $AMCE$ in a simpler way, linearly regressing $Y_{ijk}$ on an intercept $W_{ijkl}$, the latter of which is simply $D_l - 1$ dummy variables for the attribute of interest; from this regression, we obtain $\hat{\beta}_1$ as the difference in the average choice probabilities between the treatment ($T_{ijkl} = t_1$) and control ($T_{ijkl} = t_0$) groups.

Diagnostic tests as proposed by Hainmueller et al. (2013) will be conducted to check the validity of the assumptions made. For carryover effects, $AMCE$s will be estimated separately for each of the $K$ rounds of choices and look for differences. For profile order effects, I assess $AMCE$s depending on whether the attribute occurs in the first or the second profile in a given choice task. For randomization, I conduct balance tests. For attribute order effects, I assess whether $AMCE$ depends on the order in which the respective attribute appears in the conjoint table.

\newpage
\section{Conclusion}
Why don't voters punish corruption in elections under optimal informational conditions? I argue that the existing literature has found no conclusive answer to this puzzle, mainly because of assuming a retrospective voting model. I apply a different theoretical perspective on voters' decision-making to the case of corruption voting and combine prospective cost evaluations with the relative-issue salience of corruption to explain persistent corruption voting. I take a two-step approach to isolate relative-issue salience as the causal mechanism linking anticipated costs and credibility of the promise to vote choice. First, I conduct a survey experiment treating subjects with vignettes about costs and credibility of policy proposals -- varying the issues corruption, economy, and domestic security -- and compare the relative issue-salience of each of the three issues. Second, I employ a conjoint survey experiment where voters have to choose between two candidates promising policies with randomly varying levels of the attributes costs, credibility, and the issue itself. These two identification strategies enable me to add the argument to the corruption voting literature that voters do not punish corruption in elections because corruption is less salient to voters than other issues such as the economy and domestic security.

\newpage
\section{Appendix}
\subsection{Experiment 1: Treatment Vignettes} \label{sec:sec1}
\textbf{Credible-costly-corruption:}
“Imagine there is a general election upcoming. A political candidate running for office makes the following proposal to fight corruption: They want to set up an anti-corruption commission investigating and prosecuting corruption in the country. This will cost 2bn USD and will be financed from increasing income taxes. Anti-corruption commissions have been shown by experts to perform very well in decreasing the general level of corruption in numerous countries around the world."

\textbf{Credible-costly-economy:}
“Imagine there is a general election upcoming. A political candidate running for office makes the following proposal to increase economic growth: They want to reduce business tax to increase investment and thus economic prosperity. This will cost 2bn USD and will be financed from increasing income taxes. This policy has been shown by experts to perform very well in increasing economic growth and the general state of the economy in numerous countries around the world."

\textbf{Credible-costly-security:}
“Imagine there is a general election upcoming. A political candidate running for office makes the following proposal to increase domestic security: They want to increase the number of police to better fight crime and ensure border security. This will cost 2bn USD and will be financed by increasing income taxes. This policy has been shown by experts to perform very well in increasing domestic security and reducing crime in numerous countries around the world."


\textbf{Credible only-corruption:}
“Imagine there is a general election upcoming. A political candidate running for office makes the following proposal to fight corruption: They want to set up an anti-corruption commission investigating and prosecuting corruption in the country. Anti-corruption commissions have been shown by experts to perform very well in decreasing the general level of corruption in numerous countries around the world."

\textbf{Credible only-economy:}
“Imagine there is a general election upcoming. A political candidate running for office makes the following proposal to increase economic growth: They want to reduce business tax to increase investment and thus economic prosperity. This policy has been shown by experts to perform very well in increasing economic growth and the general state of the economy in numerous countries around the world."

\textbf{Credible only-security:}
“Imagine there is a general election upcoming. A political candidate running for office makes the following proposal to increase domestic security: They want to increase the number of police to better fight crime and ensure border security. This policy has been shown by experts to perform very well in increasing domestic security and reducing crime in numerous countries around the world."

\textbf{Non-credible-costly-corruption:}
“Imagine there is a general election upcoming. A political candidate running for office makes the following proposal to fight corruption: They want to set up an anti-corruption commission investigating and prosecuting corruption in the country. This will cost 2bn USD and will be financed from increasing income taxes. Anti-corruption commissions have been shown by experts to have no effect at all on decreasing the general level of corruption in numerous countries around the world."

\textbf{Non-credible-costly-economy:}
“Imagine there is a general election upcoming. A political candidate running for office makes the following proposal to increase economic growth: They want to reduce business tax to increase investment and thus economic prosperity. This will cost 2bn USD and will be financed from increasing income taxes. This policy has been shown by experts to have no effect at all on increasing economic growth and the general state of the economy in numerous countries around the world."

\textbf{Non-credible-costly-security:}
“Imagine there is a general election upcoming. A political candidate running for office makes the following proposal to increase domestic security: They want to increase the number of police to better fight crime and ensure border security. This will cost 2bn USD and will be financed by increasing income taxes. This policy has  been shown by experts to have no effect at all on increasing domestic security and reducing crime in numerous countries around the world."

\textbf{Non-credible only-corruption:}
“Imagine there is a general election upcoming. A political candidate running for office makes the following proposal to fight corruption: They want to set up an anti-corruption commission investigating and prosecuting corruption in the country. Anti-corruption commissions have been shown by experts to have no effect at all on decreasing the general level of corruption in numerous countries around the world."

\textbf{Non-credible only-economy:}
“Imagine there is a general election upcoming. A political candidate running for office makes the following proposal to increase economic growth: They want to reduce business tax to increase investment and thus economic prosperity. This policy has been shown by experts to have no effect at all on increasing economic growth and the general state of the economy in numerous countries around the world."

\textbf{Non-credible only-security:}
“Imagine there is a general election upcoming. A political candidate running for office makes the following proposal to increase domestic security: They want to increase the number of police to better fight crime and ensure border security. This policy has been shown by experts to have no effect at all on increasing domestic security and reducing crime in numerous countries around the world."

\textbf{Control-corruption:}
“Imagine there is a general election upcoming. During the election campaign, a political candidate running for office makes a proposal to fight corruption."

\textbf{Control-economy:}
“Imagine there is a general election upcoming. During the election campaign, a political candidate running for office makes the following proposal to increase economic growth."

\textbf{Control-security:}
“Imagine there is a general election upcoming. During the election campaign, a political candidate running for office makes the following proposal to increase domestic security."

\textbf{Placebo:}
“Just recently, the violinist Hilary Hahn gave a concert in Carnegie Hall in New York City. The performance has been celebrated by professional critics throughout New York's culture scene."

\subsection{Experiment 1: Outcome Measures}
\textbf{Relative issue-salience:}\\
(1) “Please rank the issues corruption, economy, domestic security, environment, education, and social equality according to the personal importance that they have to you, where 1 is the most important and 6 the least important."\\
(2) “Imagine you are planning the federal budget for the upcoming budgetary year. Please allocate percentages of money that should be spent on each issue."\footnote{Thanks to Kevin Arceneaux for this idea.}

\subsection{Experiment 1: Prognostic Pre-Treatment Covariates}
\textbf{Party Identification (Lupu 2012):}\\
“Independent of which party you usually vote for, is there a political party with which you identify?"\\
Respondents who answer positively will be asked: “With which political party do you identify?" as well as\\
“How strongly do you identify with this party on a scale from 0-10, where 10 is the strongest?"

\subsection{Experiment 2: Conjoint Table}
This model has been adopted from Hainmueller et al. (2013, 6). It is just one realization of two profiles presented to the respondent with different randomized attributes.
\begin{table}[!ht]
	\caption{Please read the descriptions of the potential candidates carefully. Then, please indicate which of the two candidates you would personally prefer to see elected in the next election.} \label{tab:tab2}
\begin{tabular}{ccc}
	\toprule
	& Candidate 1 offering policy & Candidate 2 offering policy \\
	\midrule
	Issue & Anti-Corruption & Economic \\
	Costs & No costs & 2bn USD from raising taxes \\
	Credibility & \makecell{Has been shown to perform well in \\ decreasing corruption} & \makecell{Has been shown to perform well in \\ increasing economic growth} \\
	\bottomrule
	\end{tabular}
\end{table}

\subsection{Experiment 2: Outcome Measures}
\textbf{``Forced-choice'' design (Hainmueller et al. 2013):}\\
“If you had to choose between them, which of these two candidates would you vote for in the upcoming election?"\\
\textbf{Rating design:}\\
``On a scale from 1 to 7, where 1 indicates absolutely no support for the candidate and 7 the highest support, how would you rate candidate 1?" and “Using the same scale, how would you rate candidate 2?"

\subsection{Estimation Approach Ordered Logit Model}
Define the event of interest as observing a particular score or less. The odds of observing this event are of the form

\begin{equation*}
\theta_j = \frac{Pr(score \leq j)}{Pr(score > j)}.
\end{equation*}
The ordinal logistic model for a single treatment variable is then

\begin{equation*}
ln(\theta_i) = \alpha_i + \beta X
\end{equation*}
This model will be estimated for each of the three issues separately, given treatment received as $X$. This approach is superior to a simple comparison of group means across treatment groups, since we can include additional covariates to increase precision of the estimates, as well as obtain information relating to statistical inference. In addition, it captures the ordinal nature of the outcome variable and does not treat it as a continuous variable, as a standard OLS model would do.\footnote{The results would be identical, however.}

\subsection{Estimation Assumptions in the Conjoint}
(1) \textbf{Stability and no carryover effects}: Potential outcomes always take on the same value as long as all the profiles in the same choice task have identical sets of attributes.\footnote{Note that this assumption might not always be satisfied; there are solutions as well as ways to test whether this assumption is met, however.}\\
(2) \textbf{No profile-order effects}: Ordering of profiles within a choice task does not affect responses (this can again be tested).\\
(3) \textbf{Randomization of the profiles}: Attributes of each profile are randomly generated for each of the $K$ choice tasks; potential outcomes are then statistically independent of the profiles; this assumption is met by design.\\
(4) \textbf{Completely independent randomization}: The attributes of interest are not restricted to take on particular levels depending on the values of the other attributes or profiles. All possible combinations of levels of attributes as well as profiles are, in theory, possible in this project.

\subsection{Projected Schedule of Work}
\begin{table}[H]
	\caption{Projected Schedule of Work}
	\begin{tabularx}{\textwidth}{ll}
	\toprule
	Activity & Start Date \\
	\midrule
	Funding Applications (College, Department, CESS) & June 2018 \\
	Programming of Pilot, Baseline, Experiments & July 2018 \\
	CUREC Application (IRB approval) & July 2018 \\
	Pre-registration of pilot and experiments on EGAP & September 2018 \\
	Pilot Tests & October 2018 \\
	Conduct of Experiments & November 2018 \\
	Data Analysis & December 2018 \\
	Write-Up & January 2019 \\
	Submission of Thesis & April 2019 \\
	\bottomrule
	\end{tabularx}
\end{table}

\newpage
\begin{thebibliography}{56}
\addcontentsline{toc}{section}{Bibliography}
\singlespacing

\bibitem{acharya}
Acharya, Avidit, Matthew Blackwell, and Maya Sen. 2016. ``Explaining Causal Findings Without Bias: Detecting and Assessing Direct Effects'' \textit{American Political Science Review} 110(3):512-529.

\bibitem{aisch}
Aisch, Gregor, and Alicia Parlapiano. 2017. “What Do You Think Is the Most Important Problem Facing This Country Today?" \textit{New York Times online}, February 27, 2017. \url{https://www.nytimes.com/interactive/2017/02/27/us/politics/most-important-problem-gallup-polling-question.html}

\bibitem{anduiza}
Anduiza, Eva, Aina Gallego, and Jordi Muñoz. 2013. “Turning a Blind Eye: Experimental Evidence of Partisan Bias in Attitudes Toward Corruption." \textit{Comparative Political Studies} 46(12):1664-1692. 

\bibitem{angrist}
Angrist, Joshua, and Jörn-Steffen Pischke. 2010. \textit{Mostly Harmless Econometrics. An Empiricist's Coompanion}. Princeton: Princeton University Press.

\bibitem{ansolabehere}
Ansolabehere, Stephen, and M. Socorro Puy. 2018. “Measuring Issue-Salience in Voters' Preferences." \textit{Electoral Studies} 51:103-114.

\bibitem{arceneaux}
Arceneaux, Kevin, and Martin Johnson. 2013. \textit{Changing Minds or Changing Channels? Partisan News in an Age of Choice.} Chicago, IL: The University of Chicago Press.

\bibitem{ashworth}
Ashworth, Scott. 2012. “Electoral Accountability: Recent Theoretical and Empirical Work." \textit{Annual Review of Political Science} 15:183-201.

\bibitem{azfar}
Azfar, Omar, and William R. Nelson, Jr. 2007. “Transparency, Wages, and the Separation of Powers: An Experimental Analysis of Corruption." \textit{Public Choice} 130(3):471-493.

\bibitem{bagenholm}
Bagenholm, Andreas. 2013. “Throwing the Rascals Out? The Electoral Effects of Corruption Allegations and Corruption Scandals in Europe 1981-2011." \textit{Crime, Law and Social Change} 60(5):595-609.

\bibitem{baird}
Baird, Sarah, Craig McIntosh, and Berk Özler. 2011. “Cash or Condition? Evidence from a Cash Transfer Experiment." \textit{The Quarterly Journal of Economics} 126(4):1709-1753.

\bibitem{barro}
Barro, Robert. 1973. “The Control of Politicians: An Economic Model." \textit{Public Choice} 14:19-42.

\bibitem{berinsky}
Berinsky, Adam J., Gregory A. Huber, and Gabriel S. Lenz. 2012. “Evaluating Online Labor Markets for Experimental Research: Amazon.com's Mechanical Turk." \textit{Political Analysis} 20(3):351-368.

\bibitem{botero}
Botero, Sandra, Rodrigo C. Cornejo, Laura Gamboa, Nara Pavao and David. W. Nickerson. 2015. “Says Who? An Experiment on Allegations of Corruption and Credibility of Sources." \textit{Political Research Quarterly} 68(3):493-504.

\bibitem{broockman}
Broockman, David E., Joshua L. Kalla, and Jasjeet S. Sekhon. 2017. “The Design of Field Experiments With Survey Outcomes: A Framework for Selecting More Efficient, Robust, and Ethical Designs." \textit{Political Analysis} 25:435-464.

\bibitem{chang}
Chang, Eric C., Miriam A. Golden, and Seth J. Hill. 2010. “Legislative Malfeasance and Political Accountability." \textit{World Politics} 62(2):177-220.

\bibitem{charron}
Charron, Nicholas, and Andreas Bagenholm. 2016. “Ideology, Party Systems and Corruption Voting in European Democracies." \textit{Electoral Studies} 41:35-49.

\bibitem{coppock18}
Coppock, Alexander. 2018. “Generalizing From Survey Experiments Conducted on Mechanical Turk: A Replication Approach." \textit{Political Science Research and Methods}:1-16.

\bibitem{devries}
de Vries, Catherine E., and Hector Solaz. 2017. “Electoral Consequences of Corruption." \textit{Annual Review of Political Science} 20:391-408.

\bibitem{duch08}
Duch, Raymond M., and Randy T. Stevenson. 2008. \textit{The Economic Vote: How Political and Economic Institutions Condition Election Results}. New York: Cambridge University Press.

\bibitem{duch16}
Duch, Raymond M., Pablo Beramendi, and Akitaka Matsuo. 2016. “Comparing Modes and Samples in Experiments: When Lab Subjects Meet Real People." Unpublished manuscript.

\bibitem{eggers14}
Eggers, Andrew C. and Arthur Spirling. 2014. “Guarding the Guardians: Partisanship, Corruption and Delegation in Victorian Britain." \textit{Quarterly Journal of Political Science} 9(3):337-370.

\bibitem{eggers17}
Eggers, Andrew C. and Arthur Spirling. 2017. “Incumbency Effects and the Strength of Party Preferences: Evidence from Multiparty Electoins in the United Kingdom." \textit{Journal of Politics} 79(3):903-920.

\bibitem{elinder}
Elinder, Mikael, Henrik Jordahl, and Panu Poutvaara. 2015. “Promises, Policices and Pocketbook Voting." \textit{European Economic Review} 75:177-194.

\bibitem{fearon}
Fearon, James D. 1999. Electoral Accountability and the Control of Politicians: Selecting Good Types versus Sanctioning Poor Performance. In \textit{Democracy, Accountability, and Representation}, ed. Adam Przeworski, Susan C. Stokes, and Bernard Manin, 55-97. Cambridge: Cambridge University Press.

\bibitem{feltovich}
Feltovich, Nick, and Francesco Giovannoni. 2015. “Selection vs. Accountability: An Experimental Investigation of Campaign Promises in a Moral-Hazard Environment." \textit{Journal of Public Economics} 126:39-51.

\bibitem{ferejohn}
Ferejohn, John. 1986. “Incumbent Performance and Electoral Control." \textit{Public Choice} 50(1):5-25.

\bibitem{ferraz08}
Ferraz, Claudio, and Frederico Finan. 2008. “Exposing Corrupt Politicians: The Effects of Brazil's Publicly Released Audits On Electoral Outcomes." \textit{Quarterly Journal of Economics} 123(2):703-745.

\bibitem{ferraz11}
Ferraz, Claudio, and Frederico Finan. 2011. “Electoral Accountability and Corruption: Evidence from the Audits of Local Governments." \textit{American Economic Review} 101(4):1274-1311.

\bibitem{fiorina}
Fiorina, Morris P. 1981. \textit{Retrospective Voting in American National Elections}. New Haven: Yale University Press.

\bibitem{gallup}
Gallup News. 2018. “Most Important Problem". In \textit{In Depth: Topics A to Z}. \url{http://news.gallup.com/poll/1675/most-important-problem.aspx}

\bibitem{gg}
Gerber, Alan S., and Donald P. Green. 2012. \textit{Field Experiments. Design, Analysis, and Interpretation}. New York: W.W. Norton.

\bibitem{green04}
Green, Donald, Bradley Palmquist, and Eric Schickler. 2004. \textit{Partisan Hearts and Minds. Political Parties and the Social Identities of Voters}. New Heaven: Yale University Press.

\bibitem{hainmueller}
Hainmueller, Jens, Daniel J. Hopkins, and Teppei Yamamoto. 2014. “Causal Inference in Conjoint Analysis: Understanding Multidimensional Choices via Stated Preference Experiments." \textit{Political Analysis} 22(1):1-30.

\bibitem{healymal}
Healy, Andrew, and Neal Malhotra. 2013. “Retrospective Voting Reconsidered." \textit{Annual Review of Political Science} 16:285-306.

\bibitem{holland}
Holland, Paul W. 1986. “Statistics and Causal Inference." \textit{Journal of the American Statistical Association} 81(396):945-960.

\bibitem{huff}
Huff, Connor, and Dusting Tingley. 2015. “Who Are These People? Evaluating the Democraphic Characteristics and Political Preferences of MTurk Survey Respondents." \textit{Research \& Politics} 1:1-12.

\bibitem{imai}
Imai, Kosuke, Luke Keele, Dusting Tingley, and Teppei Yamamoto. 2011. “Unpacking the Black Box of Causality: Learning about Causal Mechanisms from Experimental and Observational Studies." \textit{American Political Science Review} 105(4):765-789.

\bibitem{keele}
Keele, Luke. 2015. “The Statistics of Causal Inference: A View from Political Methodology." \textit{Political Analysis} 23(3):313-335.

\bibitem{key}
Key, Valdimer O., Jr. 1966. \textit{The Responsible Electorate}. New York: Vintage.

\bibitem{klasnja13}
Klasnja, Marko, and Joshua A. Tucker. 2013. “The Economy, Corruption, and the Vote: Evidence from Experiments in Sweden and Moldova." \textit{Electoral Studies} 32(3):536-543.

\bibitem{klasnja16}
Klasnja, Marko, Joshua A. Tucker, and Kevin Deegan-Krause. 2016. “Pocketbook vs. Sociotropic Corruption Voting." \textit{British Journal of Political Science} 46(1):67-94.

\bibitem{klasnja17}
Klasnja, Marko, Andrew T. Little, and Joshua A. Tucker. 2018. “Political Corruption Traps." \textit{Political Science Research Methods} 6(3):413-428.

\bibitem{krause}
Krause, Stefan, and Fabio Méndez. 2009. “Corruption and Elections: An Empirical Study for a Cross-Section of Countries." \textit{Economics \& Politics} 21(2):179-200.

\bibitem{lockerbie}
Lockerbie, Brad. 2008. \textit{Do Voters Look to the Future? Economics and Elections}. Albany: SUNY.

\bibitem{lupu}
Lupu, Noam. 2013. “Party Brands and Partisanship: Theory with Evidence from a Survey Experiment in Argentina." \textit{American Journal of Political Science} 57(1):49-64.

\bibitem{manin}
Manin, Bernard, Adam Przeworski, and Susan C. Stokes. 1999. Introduction. In \textit{Democracy, Accountability, and Representation}, ed. Adam Przeworski, Susan C. Stokes, and Bernard Manin, 1-26. Cambridge: Cambridge University Press. 

\bibitem{manzetti}
Manzetti, Luigi, and Carole J. Wilson. 2007. “Why Do Corrupt Politicians Maintain Public Support?" \textit{Comparative Political Studies} 40(8):949-970.

\bibitem{miller}
Miller, Joanne M., Jon A. Krosnick, and Leandre R. Fabrigar. 2016. The Origins of Policy Issue Salience: Personal and National Importance Impact on Behaviroal, Cognitive, and Emotional Issue Engagement. In \textit{Political Psychology: New Explorations}, ed. Jon A. Krosnick, I-Chant A. Chiang, and Tobias H. Stark, 125-171. New York: Taylor and Francis.

\bibitem{peters}
Peters, John G., and Susan Welch. 1980. “The Effects of Charges of Corruption on Voting Behavior in Congressional Elections." \textit{The American Political Science Review} 74(3):697-708.

\bibitem{samii}
Samii, Cyrus. 2016. “Causal Empiricism in Quantitative Research." \textit{Journal of Politics} 78(3):941-955.

\bibitem{schleiter14}
Schleiter, Petra, and Alisa M. Voznaya. 2014. “Party System Competitiveness and Corruption." \textit{Party Politics} 20(5):675-686.

\bibitem{schleiter16}
Schleiter, Petra, and Alisa M. Voznaya. 2016. “Party System Institutionalization, Accountability and Governmental Corruption." \textit{British Journal of Political Science} 48(2):315-342.

\bibitem{strezhnev}
Strezhnev, Anton, Jens Hainmueller, Daniel J. Hopkins, and Teppei Yamamoto. 2014. “Conjoint Survey Design Tool: Software Manual." Unpublished manuscript.

\bibitem{tavits}
Tavits, Margit. 2007. “Clarity of Responsibility and Corruption." \textit{American Journal of Political Science} 51(1):218-229.

\bibitem{transparency}
Transparency International. 2018. “Corruption Perceptions Index 2017" \textit{Surveys}, 21 February 2018. \url{https://www.transparency.org/news/feature/corruption_perceptions_index_2017}

\bibitem{winters}
Winters, Matthew S., and Rebecca Weitz-Shapiro. 2013. “Lacking Information or Condoning Corruption: When Will Voters Support Corrupt Politicians?" \textit{Journal of Comparative Politics} 45(4):418-436.

\end{thebibliography}


\end{document}
